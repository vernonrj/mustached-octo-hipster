\section{Design}
This project is about blah.... blah blah blah

\subsection{Alpha Predictor Design}
The Alpha Predictor uses global prediction, and local prediction. The global prediction is indexed by a global path history into an array of saturating counters. The local prediction is indexed by PC into a local history table; the result of the local history table is then used to index into an array of saturating counters.

\subsection{Return Stack Design}
The return stack is an array with an index. There is no validity checking on the return stack. 

\subsection{Branch Target Buffer Design}

\subsection{Relative vs Absolute}
Two caches are utilized for the Branch Target Buffer. A larger PC-relative cache is used for targets that are less than 1024 absolute bytes away from the instruction. This means that only 11 bits are needed for the data section of the PC-relative cache. This, combined with a 25-bit tag size, a four-way cache and 4 bits of LRU overhead per line, gives a total line size of 148 bits. 148 bits per line, with 128 lines, gives a cache size of 18,944 bits, or 18.5Kbits.

The absolute cache directly uses the lower-order 6 bits of the address to index into a table. It has a data size of 32-bits with a 26-bit tag size. With a four-way cache and 4 bits of LRU overhead per line, the total line size is 236 bits per line. 236 bits per line, with 64 lines, gives a cache size of 15104 bits, or 14.75Kbits.

This means that the relative and absolute caches use 34048 bits, or 33.25Kbits. 
